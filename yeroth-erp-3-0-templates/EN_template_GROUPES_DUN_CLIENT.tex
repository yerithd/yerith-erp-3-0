\documentclass[10pt,YEROTHPAPERSPEC]{article} % use larger type; default would be 10pt
\NeedsTeXFormat{LaTeX2e}
\makeindex

%---------------------------- PACKAGE INCLUSION -------------------------------
% This group renders characters clearer and more precise

\RequirePackage[bitstream-charter,cal,expert]{mathdesign}
\RequirePackage{charter}
\RequirePackage{helvet}
\RequirePackage{makeidx}
\RequirePackage{latexsym}

\usepackage[french]{babel}

\usepackage{geometry}
\geometry{YEROTHPAPERSPEC,
		  %showframe=true,
		  top=3.5em,
		  left=3em,
		  right=3em,
		  bottom=3.39em
		  }
		  
\usepackage{graphicx}
\usepackage{adjustbox}
\usepackage{xspace}
\usepackage[parfill]{parskip} % Activate to begin paragraphs with an empty line rather than an indent
\usepackage{paralist} % very flexible & customisable lists (eg. enumerate/itemize, etc.)
\usepackage{listings} % for lstset definitions
\usepackage{url}
\usepackage{subfig} % make it possible to include more than one captioned figure/table in a single float
\usepackage{epsfig}
\usepackage{gensymb}
\usepackage{textcomp}
\usepackage{booktabs}

\usepackage{amsmath}

\usepackage[table]{xcolor}
\definecolor{yerothColorYellow}{RGB}{254, 254, 0}
\definecolor{yerothColorOrange}{RGB}{242, 161, 0}   
\definecolor{yerothColorBlue}{RGB}{77 , 93 , 254}
\definecolor{yerothColorRed}{RGB}{254, 48 , 48}
\definecolor{yerothColorGray}{RGB}{198, 198, 198}
\definecolor{yerothColorDarkgray}{RGB}{60, 60 , 60}
\definecolor{yerothColorIndigo}{RGB}{83, 0, 125}
\definecolor{yerothColorGreen}{RGB}{2  , 160, 70}
\definecolor{forestgreen}{RGB}{2,160,70}    
\definecolor{mediumblue}{RGB}{7,43,205}    
\definecolor{firebrickred}{RGB}{178,34,34}
\definecolor{listingray}{gray}{0.9}
\definecolor{lbcolor}{rgb}{0.9,0.9,0.9}
\definecolor{darkgreen}{rgb}{0,0.35,0}
\definecolor{medgreen}{rgb}{0,0.5,0}
\definecolor{lightgreen}{rgb}{0.5,0.7,0.5}
\definecolor{pmcolour}{rgb}{0.5,0.7,0.5}
\definecolor{medgrey}{rgb}{0.6,0.6,0.6}
\definecolor{purplish}{rgb}{0.4,0,0.6}
\definecolor{brightred}{rgb}{1,0.2,0.2}

\usepackage{hyperref}
\hypersetup{
colorlinks,
pagebackref,
citecolor=medgreen,
linkcolor=purplish,
breaklinks,
pdftex,
bookmarks,
plainpages=false,
pdftitle={CLIENT GROUPS FOR CLIENT YEROTHCLIENT, YEROTHENTREPRISE},
pdfauthor={YEROTHENTREPRISE (YEROTHNOMUTILISATEUR)}
}

% format two pieces of text, one left aligned and one right aligned
\newcommand{\headerrow}[2]
{\begin{tabular*}{\linewidth}{l@{\extracolsep{\fill}}r}
	#1 &
	#2 \\
\end{tabular*}}

\newcommand{\emphbold}[1]{\textbf{\emph{#1}}\xspace}

\usepackage{lastpage}
\usepackage{fancyhdr}
\pagestyle{fancy}
\renewcommand{\headrulewidth}{0pt}
\fancyhf{}

\fancypagestyle{plain}{% copies "fancy" over "plain"
  \fancyfoot[C]{\thepage}% you can add edits that won't affect "fancy" but only "plain"
}

\fancypagestyle{OnlyFirstPage}{%
	\lhead{}
	\rhead{}
}

\rhead{}
\lhead{}
\lfoot{}
\rfoot{}
\cfoot{\thepage\ DE \pageref{LastPage}}

\begin{document}

\bigskip

\headerrow
	{\emphbold{YEROTHENTREPRISE}}
	{\emph{\textbf{Tax payer registration n\textsuperscript{o}:} YEROTHCONTRIBUABLENR}}
\headerrow
	{\emphbold{YEROTHACTIVITESENTREPRISE}}
	{\emph{\textbf{Bank account n\textsuperscript{o}:} YEROTHCOMPTEBANCAIRENR,}}
\headerrow
	{\emphbold{Email: YEROTHEMAIL}}
	{\emph{YEROTHAGENCECOMPTEBANCAIRE}}
\headerrow
	{\emphbold{Tel.: YEROTHTELEPHONE}}
	{}
\headerrow
	{\emphbold{PO Box: YEROTHBOITEPOSTALE YEROTHVILLE}}
	{}
	
\hrule

\headerrow
	{}
	{\textbf{YEROTHDATE}}

\vspace*{0.9cm}

\headerrow
{}
{\textbf{YEROTHCLIENT}}
\headerrow
{}
{PO Box: CLIENTYEROTHPOBOX CLIENTYEROTHCITY}
\headerrow
{}
{Email: CLIENTYEROTHMAIL}
\headerrow
{}
{Phone: CLIENTYEROTHPHONE}

\section*{YEROTHSUBJECT}

\textbf{Printed by:} YEROTHNOMUTILISATEUR\\
\textbf{FILE (this) GENERATION Site:} YEROTHSUCCURSALE\\
\textbf{Time of print:} YEROTHHEUREDIMPRESSION

\vspace{0.3cm} 

%\begin{table*}[!htbp]
%\resizebox{\textwidth}{!}{
%\centering
%\begin{tabular}{|l|l|c|} \hline
% & & 				\\
%DESIGNATION 		&
%GROUP REFERENCE  	&
%MAX MEMBERS 	\\
% & &  				\\ \hline
%\hline
%YEROTHGROUPEDUNEMPLOYEDESIGNATION 		& 
%YEROTHGROUPEDUNEMPLOYEGROUPEDEPAIE 		& 
%YEROTHGROUPEDUNEMPLOYENOMBREDEMEMBRES 	\\ \hline

%\end{tabular}}
%\end{table*}
