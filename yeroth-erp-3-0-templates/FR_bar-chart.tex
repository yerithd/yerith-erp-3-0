\documentclass[11pt,YEROTHPAPERSPEC]{article} % use larger type; default would be 10pt
\NeedsTeXFormat{LaTeX2e}
\makeindex

%---------------------------- PACKAGE INCLUSION -------------------------------
% This group renders characters clearer and more precise

\RequirePackage[bitstream-charter,cal,expert]{mathdesign}
\RequirePackage{charter}
\RequirePackage{helvet}
\RequirePackage{makeidx}
\RequirePackage{latexsym}

\usepackage{geometry}
\geometry{YEROTHPAPERSPEC,
		  top=3.5em,
		  left=3em,
		  right=3em,
		  bottom=3.39em
		  }
		  
\usepackage{graphicx}
\usepackage{adjustbox}
\usepackage{xspace}
\usepackage[parfill]{parskip} % Activate to begin paragraphs with an empty line rather than an indent
\usepackage{paralist} % very flexible & customisable lists (eg. enumerate/itemize, etc.)
\usepackage{listings} % for lstset definitions
\usepackage{url}
\usepackage{subfig} % make it possible to include more than one captioned figure/table in a single float
\usepackage{epsfig}
\usepackage{gensymb}
\usepackage{textcomp}
\usepackage{booktabs}

\usepackage{amsmath}

\usepackage{hyperref}
\hypersetup{
colorlinks,
pagebackref,
citecolor=medgreen,
linkcolor=purplish,
breaklinks,
pdftex,
bookmarks,
plainpages=false,
pdftitle={Diagramme à bandes r\'epr\'esentant "YEROTHBARCHARTTITLE" du YEROTHVENTESDEBUT au YEROTHVENTESFIN (YEROTHENTREPRISE)},
pdfauthor={YEROTHENTREPRISE (YEROTHUTILISATEUR)}
}

\usepackage{databar}

\usepackage{xcolor}

\definecolor{forestgreen}{RGB}{2,160,70}
\definecolor{mediumblue}{RGB}{27,93,255}
\definecolor{firebrickred}{RGB}{178,34,34}
\definecolor{listingray}{gray}{0.9}
\definecolor{lbcolor}{rgb}{0.9,0.9,0.9}
\definecolor{darkgreen}{rgb}{0,0.35,0}
\definecolor{medgreen}{rgb}{0,0.5,0}
\definecolor{lightgreen}{rgb}{0.5,0.7,0.5}
\definecolor{medgrey}{rgb}{0.6,0.6,0.6}
\definecolor{purplish}{rgb}{0.4,0,0.6}
\definecolor{brightred}{rgb}{1,0.2,0.2}
\definecolor{yerothbackground}{RGB}{226,226,226}

\DTLloaddb{yerothbardb}{YEROTHCSVFILE}

YEROTHDTLSETBARCOLOR

% format two pieces of text, one left aligned and one right aligned
\newcommand{\headerrow}[2]
{\begin{tabular*}{\linewidth}{l@{\extracolsep{\fill}}r}
	#1 &
	#2 \\
\end{tabular*}}

\newcommand{\emphbold}[1]{\textbf{\emph{#1}}\xspace}


\begin{document}

%\pagecolor{yerothbackground}

\bigskip

\headerrow
	{\emphbold{YEROTHENTREPRISE}}
	{\emph{\textbf{N\textsuperscript{o} de contribuable:} YEROTHCONTRIBUABLENR}}
\headerrow
	{\emphbold{YEROTHACTIVITESENTREPRISE}}
	{\emph{\textbf{N\textsuperscript{o} de compte bancaire:} YEROTHCOMPTEBANCAIRENR,}}
\headerrow
	{\emphbold{\'Email: YEROTHEMAIL}}
	{\emph{YEROTHAGENCECOMPTEBANCAIRE}}
\headerrow
	{\emphbold{T\'el.: YEROTHTELEPHONE}}
	{}
\headerrow
	{\emphbold{B.P.: YEROTHBOITEPOSTALE YEROTHVILLE}}
	{}
	
\hrule

\headerrow
	{}
	{\textbf{YEROTHDATE}}

\section*{Diagramme à bandes r\'epr\'esentant "YEROTHBARCHARTTITLE" (en pourcentage \%).}
\textbf{G\'en\'erer par:} YEROTHUTILISATEUR\\
\textbf{SUCCURSALE DE GÉNÉRATION DU FICHIER.:} YEROTHSUCCURSALE\\
\textbf{Heure de g\'en\'eration:} YEROTHHEUREGENERATION\\
\textbf{P\'eriode:} YEROTHVENTESDEBUT -- YEROTHVENTESFIN

\begin{figure}[!htbp]
\setlength{\DTLbaroutlinewidth}{1pt}
\centering
\DTLbarchart{variable=\theQuantity,barlabel=\theName,%
upperbarlabel=\theQuantity}{yerothbardb}{%
\theQuantity=Total,\theName=Nom}
\caption{YEROTHBARCHARTTITLE.}
\end{figure}

YEROTHCHARTFIN

\end{document}
